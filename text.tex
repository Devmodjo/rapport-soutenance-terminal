\documentclass[english,12pt,a4paper]{report}
\usepackage[utf8]{inputenc}
\usepackage[T1]{fontenc}
\usepackage{amsmath}
\usepackage{amssymb}
\usepackage{graphicx}
\usepackage[top=1.5cm,bottom=2cm,left=2cm,right=2cm]{geometry}
\usepackage{array}
\usepackage{multirow}
\usepackage{multicol}
\usepackage{enumerate}
\usepackage{layout}
\usepackage{dirtytalk}
\usepackage{setspace}
\usepackage{sectsty}
\usepackage{float}
\usepackage{gensymb}
\usepackage{tikz}
\usepackage{enumitem}
\usepackage{listings}
\usepackage{comment}
\usepackage{caption}
\usepackage{subcaption}
\usepackage{fancyhdr}
\usepackage{blindtext}
\usepackage{parskip}
\usepackage{babel}
\usepackage{hyperref}
\setlength{\parindent}{4em}
\setlength{\parskip}{1em}
\chapterfont{\centering}
\pagestyle{fancy}
\fancyhf{}
\fancyheadoffset{0cm}
\renewcommand{\headrulewidth}{0pt}
\renewcommand{\footrulewidth}{0pt}
\fancyhead[R]{\thepage}
\fancypagestyle{plain}{\fancyhf{}\fancyhead[R]{\thepage}}
\begin{document}
\begin{spacing}{1.5}
\begin{titlepage}
\begin{center}
\begin{center}
\begin{tabular}{c c c}
	REPUBLIQUE DU CAMEROUN &             & REPUBLIC OF CAMEROON\\
	Paix-Travail-Patrie &             & Peace-Work-Fatherland\\
	\vspace{1cm} 
	MINESEC &             & MINESEC\\
	DELEGATION DEPARTEMENTALE  &             & MENOUA DIVISIONAL \\
	DE LA MENOUA &             & DELEGATION \\
	LYCEE CLASSIQUE DE DSCHANG &   \hspace{3cm}          & DEPARTEMENT D'INFORMATIQUE
\end{tabular}
\end{center}
\vspace{1.5cm}
{\huge\scshape\textbf{RAPPORT DE STAGE}}\\
\vspace{1.5cm}
{{\Large\scshape\textbf{THEME: UNE APPLICATION DESKTOP POUR LA GESTION DES EMPLOIES DE TEMPS SCOLAIRE}}\\{\Large Redigé et presenté \emph{par: }\\
\textbf{KAMSU MODJO\emph{(230968817)}\\} \textbf{TATANG NGUENA\emph{(231305012)}}\\}}
\vspace{1.2cm}
{\large{Ce projet soumis au collège de formation des enseignants d'informatiques du lycée classique de Dschang dans le cadre de l'obtention partielle du BACCALAUREAT ESG de la série TI(technologie de l'information)}\\}
\vspace{1cm}
{\large Supervisé \emph{par:}
\textbf{Mr. GOUFACK}\\
\large co-encadré \emph{par:}
\textbf{Mr. ELLA EMMANUEL}}
\vfill
{\itshape Mai, 2025}
\end{center}
\end{titlepage}
\cleardoublepage
\pagenumbering{roman}
\addcontentsline{toc}{chapter}{CERTIFICATION}
\chapter*{CERTIFICATION}
\hspace{1.2cm}
Nous attestons que ce projet intitulé << UNE APPLICATION DESKTOP POUR LA GESTION DES EMPLOIE DE TEMPS SCOLAIRE>> a été réalisé par \textbf{KAMSU MODJO YVAN} et \textbf{TATANG NGUENA} avec les numéro de matricule 230968817 et 231305012 de la série Technologie de l'information du département d'informatique du lycée classique de Dschang.
\begin{center}
	\vspace{0.2cm}
	\emph{Encadreur}\\
	\vspace{0.2cm}
	\textbf{Mr. GOUFACK}\\
	\vspace{0.2cm}
	Date:
	\begin{tikzpicture}
		\draw(0,0)--(4,0);
	\end{tikzpicture}\\
	\vspace{0.2cm}
	\emph{Co-Encadreur}\\
	\vspace{0.2cm}
	\textbf{Mr. ELLA EMMANUEL}\\
	\vspace{0.2cm}
	Date:
	\begin{tikzpicture}
		\draw(0,0)--(4,0);
	\end{tikzpicture}\\
	\vspace{0.5cm}
	\emph{Animateur Pédagogique}\\
	\vspace{0.2cm}
	\textbf{Mr. ELLA EMMANUEL}\\
	\vspace{0.2cm}
	Date:
	\begin{tikzpicture}
		\draw(0,0)--(4,0);
	\end{tikzpicture}
\end{center}
\addcontentsline{toc}{chapter}{ATTESTATION}
\chapter*{ATTESTATION}
\hspace{1.2cm}
C'est avec beaucoup d'enthousiasme que nous sommes les auteurs de ce projet << UNE APPLICATION DE BUREAU POUR LA GESTION DES EMPLOIES DE TEMPS SCOLAIRE >> et nous donnons la permission au lycée classique de Dschang, par l'intermédiaire du département d'informatique, de prêter ce projet à d'autres établissements scolaires ou individus à des fins de recherche académique. Nous comprenons la nature du plagiat et nous somme conscients de la politique éducative à ce sujet; nous somme convaincus que ce sujet est original et réalisé par nous au cours de nos étude pour l'obtention du BACCALAUREAT ESG en Technologie de l'information.
\begin{center}
\vspace{1cm}
\vspace{0.2cm}
\textbf{KAMSU MODJO YVAN}\\
\textbf{TATANG NGUENA}\\
\vspace{0.2cm}
Date:
\begin{tikzpicture}
	\draw(0,0)--(4,0);
\end{tikzpicture}\\
\vspace{0.2cm}
Signature:
\begin{tikzpicture}
	\draw(0,0)--(4,0);
\end{tikzpicture}
\end{center}
\phantomsection
\addcontentsline{toc}{chapter}{DEDICACE}
\chapter*{DEDICACE}
\hspace{1.2cm} 
\begin{center}
	\textbf{nous dédions ce travail à nos parents}
\end{center}
\vspace{-1.5cm}
\addcontentsline{toc}{chapter}{REMERCIEMENT}
\chapter*{REMERCIEMENT}
\vspace{-1.5cm}
\hspace{1.cm}
le mérite d'un travail ne revient pas toujours systématiquement à ses seuls acteurs. Nous ne pouvions donc pas conclure un tel ouvrage sans saluer les efforts de certaines personnes qui nous ont aidés et guidés de diverses manières. c'est pourquoi nous tenons à leur exprimer notre profonde gratitude, en particulier:
\begin{itemize}[label=\textbullet, font=\LARGE %\color{blue}
	]
	\item Le proviseur du Lycée Classique de Dschang - \textbf{Mr ATEM NDE JEAN CLAUDE} qui a permis toutes les nécessités pour notre meilleure formation.
	
	\item Le Censeur en charge de l'informatique - \textbf{Mr TEMGOUA} pour sa faculté à nous trouver du stage et à veiller qu'on reçoivent de bon enseignements.
	
	\item L'animateur Pédagogique \textbf{Mr ELLA EMMANUEL}, Un modèle à suivre, pour nous avoir aider et guider tout au long de notre formation.
	
	\item Notre encadreur  \textbf{	Mr GOUFACK }  pour ses conseils   
	
	\item Notre co-encadreur \textbf{Mr. ELLA EMMANUEL } Pour sa supervision, sa disponibilité et ses conseils lors de notre formation  et pour sa supervision 
	
	
	\item du personnel administratif, le Collège des enseignants du lycée classique de Dschang en général et les enseignants du département d'informatique en particulier;
	
	\item Tous les membres du jury qui ont pris leur temps précieux pour lire et nous aider à améliorer ce travail ; 
	
	\item 	 Nos parents, pour leur esprit de sacrifice, leur soutien financier et moral ainsi que leurs conseils et encouragements
	\item 	 à tous nos camarades de classe qui ont créé autour de nous une excellente ambiance de travail
	\item 	 nous remercions également ceux dont les noms ne sont pas mentionnés ici, pour leur aide
\end{itemize}	

\addcontentsline{toc}{chapter}{RESUME}
\chapter*{RESUME}
\hspace{1.2cm}
Dans un établissement scolaire, la gestion des emplois du temps est une tâche complexe qui nécessite une organisation rigoureuse. Un mauvais agencement des horaires peut entraîner des conflits entre les enseignants, les salles de classe et les matières enseignées.

L’objectif de ce projet est de développer une application de gestion des emplois du temps, permettant d’optimiser la planification des cours en fonction des disponibilités des professeurs, des salles et des classes. Cette application, développée en JavaFX et l'IDE Eclipse avec une base de données SQLite, vise à automatiser la création et la gestion des emplois du temps pour améliorer l’efficacité et réduire les erreurs humaines.

Ce rapport présente l’analyse du projet, les acteurs impliqués, ainsi que l’étude de faisabilité technique et organisationnelle.
\bigskip 
\bigskip
\par 

%\textbf{Mots-clés:} IMMOBIZ, application mobile, Android, immobilier
\addcontentsline{toc}{chapter}{ABSTRACT}
\chapter*{ABSTRACT}
\hspace{1.2cm}
In a school setting, timetable management is a complex task requiring rigorous organization. Poor scheduling can lead to conflicts between teachers, classrooms, and subjects taught.

The goal of this project is to develop a timetable management application to optimize class scheduling based on teacher availability, classroom availability, and class groups. This application, developed in JavaFX with an SQLite database, aims to automate timetable creation and management to improve efficiency and reduce human errors.

This report presents the project analysis, involved stakeholders, and the technical and organizational feasibility study.


\bigskip 
\bigskip
\par 

\textbf{Keywords:} IDE, JavaFX, application desktop, 
\addcontentsline{toc}{chapter}{LISTE DES ABRÉVIATIONS}
\chapter*{LISTE DES ABRÉVIATIONS}
\textbf{IDE} : Integrated Development environment 

\textbf{OS} : 	 Operating System 

\textbf{SQL} :  	Structured Query Language

\textbf{UML} : Unified Modeling Language

\textbf{CSS} : Cascading Style Sheets
\end{spacing}

\tableofcontents

\cleardoublepage
\phantomsection

\addcontentsline{toc}{chapter}{LISTE DES FIGURES}
\listoffigures

\cleardoublepage
\phantomsection

\addcontentsline{toc}{chapter}{LISTE DES TABLEAUX}
\listoftables


\addcontentsline{toc}{chapter}{INTRODUCTION GENERALE}


\chapter*{INTRODUCTION GENERALE}
\vspace{0.8cm}
\pagenumbering{arabic}

\section{contexte}
Dans un monde en constante évolution, l'organisation et la gestion efficace du temps sont devenues des enjeux majeurs pour les établissements scolaires. L'emploi du temps constitue un élément fondamental du bon fonctionnement d'un établissement, permettant de structurer les activités pédagogiques, d'optimiser l'utilisation des ressources et d'assurer un équilibre entre les contraintes des enseignants, des élèves et des infrastructures disponibles. Cependant, la planification et la gestion manuelle des emplois du temps peuvent s'avérer complexes et chronophages, entraînant des conflits d'horaire, une mauvaise allocation des salles et une difficulté à répondre aux imprévus. Pour pallier ces problématiques, l'intégration d'un système informatisé devient une nécessité afin d'automatiser et de simplifier ces processus.

C’est dans cette optique que notre projet de gestion des emplois du temps a été conçu. Il vise à fournir une solution logicielle efficace permettant aux administrateurs scolaires de créer, modifier et consulter les emplois du temps en toute simplicité. Ce système offrira également aux enseignants et aux responsables pédagogiques un accès rapide aux informations essentielles, facilitant ainsi la coordination et l’organisation des cours. À travers ce rapport, nous présenterons l’ensemble des aspects liés à la conception, au développement et à la mise en œuvre de cette application, en mettant l’accent sur les besoins des utilisateurs, les choix technologiques et les bénéfices attendus.

\section{La problématique}
Dans un établissement scolaire, la gestion des emplois du temps est un processus essentiel mais souvent complexe. Entre la disponibilité des enseignants, la répartition des salles, la prise en compte des différentes disciplines et les contraintes spécifiques de chaque niveau d’enseignement, l’élaboration d’un emploi du temps optimal devient un véritable casse-tête. Les méthodes traditionnelles, souvent basées sur des tableaux manuels ou des fichiers Excel, sont limitées et sujettes aux erreurs : chevauchements d’horaires, mauvaise allocation des ressources, difficultés d’adaptation aux imprévus (absences, changements de salle, etc.). De plus, l’absence d’un système centralisé complique l’accès et la mise à jour rapide des informations pour les différents acteurs de l’établissement. Ainsi, la question centrale que soulève ce projet est la suivante :
\textbf{Comment concevoir et mettre en place un système informatisé permettant une gestion efficace, flexible et automatisée des emplois du temps dans un établissement scolaire, tout en assurant un accès rapide et structuré aux informations pour les administrateurs, enseignants et autres parties prenantes ?} Ce projet vise donc à proposer une solution logicielle innovante permettant d’optimiser la planification des emplois du temps tout en réduisant les erreurs humaines et en améliorant la coordination entre les acteurs concernés

\section{Objectifs du projet }
 L’objectif principal de ce projet est de concevoir et développer une application informatique de gestion des emplois du temps pour un établissement scolaire, permettant une planification efficace et automatisée des cours, des enseignants et des salles
 Pour atteindre l'objectif standard, les objectifs suivants doivent être poursuivis : 
\begin{itemize}
	\item Automatiser la planification des emplois du temps afin de réduire les erreurs humaines et optimiser l’organisation des cours.
	\item Faciliter l’accès aux informations pour les différents acteurs (administrateurs, enseignants, responsables des emplois du temps) grâce à une interface intuitive et centralisée
	\item Gérer dynamiquement les imprévus tels que les absences des enseignants ou l’indisponibilité des salles, en permettant des mises à jour rapides et efficaces.
	\item Garantir une meilleure organisation des ressources (enseignants, salles, disciplines) afin d’éviter les conflits d’horaires et d’améliorer l’efficacité du système éducatif.
	\item Proposer un accès sécurisé et différencié selon le rôle de l’utilisateur (administrateur, responsable des emplois du temps, professeur).
\end{itemize}

\subsection{Objectif général du projet}

L'objectif général de cette étude est de concevoir et de créer une application de bureau pour faciliter la gestion des emploie de temps dans un établissement scolaire.

\subsection{Objectifs spécifiques}
Notre système doit être en mesure de faire ce qui suit:
\begin{itemize}
	\item Gérer les ressources : Professeurs, classes, matières et salles de cours.
	\item Éviter les conflits d’horaires en assurant une bonne répartition des cours et des salles.
	\item Offrir une interface intuitive pour que les administrateurs, enseignants et élèves puissent consulter les emplois du temps.
	\item Faciliter les modifications et mises à jour en cas de changements de dernière minute.
	\item déterminer s'il existe une disparité dans les valeurs locatives des propriétés résidentielles et commerciales dans la zone d'étude.
	
\end{itemize}


\section{Importance du projet }
La mise en place d’une application de gestion des emplois du temps revêt une importance capitale pour un établissement scolaire. En effet, la gestion manuelle des emplois du temps est souvent source d’erreurs, de conflits d’horaires et de pertes de temps considérables. Ce projet apporte donc plusieurs avantages :

\begin{itemize}
	\item \textbf{Optimisation de la gestion des emplois du temps} : L’automatisation permet d’attribuer efficacement les cours aux enseignants et aux salles tout en respectant les contraintes pédagogiques et administratives.
	\item \textbf{Réduction des erreurs humaines} : En évitant les conflits d’horaires, les doubles affectations et les oublis, l’application garantit une meilleure organisation.
	\item \textbf{Gain de temps} : La génération automatique des emplois du temps réduit le travail manuel fastidieux, permettant ainsi aux administrateurs et responsables de se concentrer sur d’autres tâches essentielles.
\end{itemize}

\section{Le plan de rédaction }

Ce travail est structuré comme suit :

\begin{itemize}
	\item 	la première section intitulée « Introduction générale » comprenant le contexte de l'étude, la problématique, les objectifs de l'étude, l'importance de l'étude, la portée de l'étude et l'aperçu de la rédaction; 
	\item 	 Le premier chapitre intitulé « Contexte et revue de la littérature », y compris le contexte, la définition des termes et la revue de la littérature; 
	\item 	 Le chapitre deux intitulé "Matériaux et méthodes" qui est le développement de tous les matériels et logiciels utilisés dans ce projet. L'aspect méthodologique comprenant une description des différents outils matériels et logiciels utilisés pour concevoir l'application et des méthodes d'analyse ; 
	\item 	le chapitre trois intitulé « Évaluation des coûts (estimation) et réalisation » qui comprend le coût estimé de notre projet et les détails sur les différents diagrammes utilisés dans ce travail; 
	\item The  Le chapitre quatre intitulé "Résultats et discussion" où les différents résultats seront affichés pour montrer les réponses à nos objectifs de recherche et la discussion sur les résultats obtenus ; 
	\item  La dernière section intitulée « Conclusion et recommandations » constituait le résumé des conclusions et recommandations. 
\end{itemize}
\end{document}